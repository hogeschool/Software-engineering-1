The SVM is made of a set of memory locations, which index is called address. The total amount of memory locations available in the SVM can be specified by the user. Each memory location can contain integer, floating point, or string values (for simplicity, as this is different from hardware architectures and standard virtual machines where you only save bytes). Furthermore, unlike normal virtual machines, the values used in the assembly language are typed, meaning that the type compatibility is checked when executing a statement, in the same fashion as in high-level programming languages where, for example, you cannot sum a \texttt{string} with an \texttt{integer}.

The virtual machine has also 4 registers \texttt{reg1}, \texttt{reg2}, \texttt{reg3}, and \texttt{reg4}, that are special memory location used by the instructions to save temporary results or to contain temporary data. The data types contained in the registers can be the same as the main memory. It has also a special register, called \textit{Program Counter}, which contains the index of the current instruction to be executed.

Unlike low-level architectures, the program you have to execute is not saved in the memory but it is kept separate (thus the memory contains only the data and not the instructions the SVM has to execute).